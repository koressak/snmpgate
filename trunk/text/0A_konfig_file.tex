\chapter{Struktura konfiguračního souboru}
\label{kap_struktura_conf_souboru}
Konfigurační soubor je samostatný XML dokument, který se skládá z několika definovaných částí.
Hlavní element je \verb|<devices>...</devices>|. Tento vymezuje všechna spravovaná a monitorovaná zařízení.

Existují dva typy zařízení - protokolová brána a jakékoliv jiné. 

\subsection*{Běžné zařízení}
Každé zařízení musí být specifikováno dle následujícího příkladu:

\begin{verbatim}
<device id="XY">
  <name>Název zařízení</name>
  <description>Popis pro lepší pochopení.</description>
  <snmpAddr>ip_adresa</snmpAddr>
  <protocolVersion>2</protocolVersion>
  <mibs>
    <mib>RFC1213-MIB.txt</mib>
    ...
  </mibs>
  <traps>
    <manager>
      <address>ip_adresa</address>
      <port>7878</port>
    </manager>
    ...
  </traps>
  <access>
    <read>public</read>
    <write>private</write>
  </access>
</device>
\end{verbatim}

Nejprve je nutné definovat identifikační číslo zařízení. Id 0 je rezervováno pro protokolovou bránu
a systém nepovolí zařazení jiného zařízení místo ní. Tímto identifikačním číslem jsou pak opatřeny
jednotlivé zprávy.

Následuje jméno a popis zařízení. To vše pro lepší uživatelský přístup administrátora při správě brány.
Je nutné definovat adresu snmp agenta (url či ip adresa) a verzi SNMP protokolu (je podporována verze 1 a 2).

Dalším prvkem jsou vyjmenované jednotlivé MIB databáze, které specifikují škálu nabízených informací. Tyto
soubory s definicí dat musejí být přítomny na stroji brány. Budou zpracovávány v transformační fázi. Pakliže
není specifikována ani jedna, je nahrána základní sada MIB databází dle knihovny net-snmp.

Element \verb|<traps>...</traps>| definuje množinu manažerů, kteří budou obesláni, pakliže nastane nějaká
asynchronní událost. Každý manažer je definován adresou a portem, na kterém poslouchá.

Posledním základním elementem je definice přístupu. Zde je možné definovat SNMP hesla (community řetězce)
pro čtení a zápis. Tyto nemají nic společného s definovavými hesly XML protokolu (viz níže). Danému manažerskému
požadavku se přiřadí takový řetězec, jaká práva mu příslušejí dle XML hesla.

\subsection*{Protkolová brána}
Samotná brána má, oproti běžným zařízením, ještě několik speciálních XML elementů, které definují podstatné
součásti chování programu. Úplný soupis elementů následuje:

\begin{verbatim}
<logFile>/tmp/snmpxmld.log</logFile>
<snmp>
  <mibPath>/usr/share/snmp/mibs/</mibPath>
  <listenPort>3111</listenPort>
</snmp>
<xml>
  <xsdPath>/tmp/</xsdPath>
  <listenPort>8888</listenPort>
  <transmitPort>2555</transmitPort>
  <access>
    <read>cteni</read>
    <write>zapis</write>
  </access>
  <protocolVersion>1</protocolVersion>
</xml>
<security>
  <key>key.pem</key>
  <certificate>cert.pem</certificate>
</security>
\end{verbatim}

Je možné specifikovat logovací soubor, nebo používat již předdefinovanou cestu.

Následující elementy specifikují části SNMP a XML modulů. V rámci SNMP lze nastavit
cestu k MIB souborům a port, na kterém jsou poslouchány asynchronní události.

XML modul je možné definovat o poznání více. Neprve cesta, kam se budou ukládat
Xsd popisné soubory. 

\verb|listenPort| specifikuje, na kterém portu bude puštěn HTTP server. Zde se očekávají
požadavky od manažerů.

\verb|transmitPort| definuje port, na který se odesílají zprávy Distribution.

Element přístupu specifikuje, stejně jako v SNMP části běžného zařízení, jaká jsou hesla pro
čtení a zápis.

Posledním elementem je definice klíče a certifikátu, které HTTP server používá při spojení s manažery.
Tento element je volitelný.

\subsection*{Příklad souboru}
Následuje příklad funkčního konfiguračního souboru o dvou monitorovaných zařízeních.

\begin{verbatim}
<?xml version="1.0" encoding="UTF-8"?>
<devices>
  <!-- Brana -->
  <device id="0">
    <name>SNMP Gate</name>
    <description>Popis brany</description>
    <snmpAddr>localhost</snmpAddr>
    <protocolVersion>2</protocolVersion>
    <mibs>
      <mib>RFC1213-MIB.txt</mib>
    </mibs>
    <traps>
      <manager>
        <address>127.0.0.1</address>
        <port>7878</port>
      </manager>
    </traps>
    <access>
      <read>public</read>
      <write>private</write>
    </access>

    <!--
    Specificke elementy brany
    -->
    <logFile>/tmp/snmpxmld.log</logFile>
    <snmp>
      <mibPath>/usr/share/snmp/mibs/</mibPath>
      <listenPort>3111</listenPort>
    </snmp>
    <xml>
      <xsdPath>/tmp/</xsdPath>
      <listenPort>8888</listenPort>
      <transmitPort>2555</transmitPort>
      <access>
        <read>cteni</read>
        <write>zapis</write>
      </access>
      <protocolVersion>1</protocolVersion>
    </xml>
  </device>

  <!-- Monitorovany notebook -->
  <device id="1">
    <name>notebook</name>
    <description>Domaci notebook</description>
    <snmpAddr>192.168.1.100</snmpAddr>
    <protocolVersion>2</protocolVersion>
    <mibs>
      <mib>RFC1213-MIB.txt</mib>
    </mibs>
    <traps>
      <manager>
        <address>192.168.1.50</address>
        <port>1048</port>
      </manager>
    </traps>
    <access>
      <read>home</read>
      <write>house</write>
    </access>
  </device>
</devices>
\end{verbatim}

