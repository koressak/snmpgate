\chapter{Uživatelská příručka protokolové brány}
\label{kap_uzivatelska_prirucka_brana}
Samotná protokolová brána funguje jako démon. Je ovládána pomocí spouštěcího skriptu, který je k
programu přiložen. Vzhledem k tomu, že za cílový operační systém je zvolen Linux, budeme se v dalším 
popisu adresářů opírat o jeho strukturu souborů.

\subsection*{Instalace}
Pro úspěšnou instalaci je nejprve nutné program zkompilovat. Pro přeložení je nutné mít v systému nainstalovány následující knihovny
(pro každou knihovnu platí, že je nutné mít hlavičkové soubory, nikoliv jenom runtime verzi).
\begin{itemize}
	\item \textbf{Xerces-C++} - verze 2.8.0 a vyšší
	\item \textbf{Xalan-C++} - verze 1.10 a vyšší
	\item \textbf{libsnmp} - verze 5.4.1
	\item \textbf{libmicrohttpd} - 0.4.0
	\item \textbf{libcurl} - 7.18.2
\end{itemize}

Zkompilovaný program zkopírujeme do libovolného adresáře (nejlépe však \verb|/usr/bin| či \verb|/usr/share/bin|). 
Spouštěcí skript umístíme do adresáře \verb|/etc/init.d/|. Je nutné jej však upravit a nastavit přesné umístění
binárního souboru, který bude spouštěn.

\subsection*{Spuštení a běh}
Samotné spuštění a běh není nikterak náročné. Jediným parametrem, který program přijímá je umístění konfiguračního souboru.
Cesta k souboru je uvedena v rámci spouštěcího skriptu. Proto jakékoliv změny je nutné promítnout i tam, aby program nahrával
aktuální konfiguraci.

Protokolovou bránu lze spouštět při startu programu, či později manuálně. Vše je ponecháno na administrátorovi systému.

\subsubsection*{MIB soubory}
Při specifikaci MIB databázových souborů v konfiguračním souboru je nutné dbát na to, aby všechny uvedené existovaly
v umístění, které je součástí konfigurace zařízení brány. Pakliže některé soubory jsou nedostupné, systém nebude moci
dané zařízení spravovat.



