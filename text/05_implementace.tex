\chapter{Implementace}
\label{kap_implementace}

%TODO: upravit tento uvodni text az po dopsani cele kapitoly.

Vlastní implementace protokolové brány je v této kapitole vysvětlena v rámci několika částí. Nejprve je popsána
struktura programu, popis tříd, jejich vztah mezi sebou a hlavní funkce, kterou plní. Poté je
vysvětlen výkonný cyklus programu krok za krokem od spuštění až po zpracování jednotlivých zpráv. Následuje seznam všech
použitých knihoven, které jsou pro chod programu nezbytné. V poslední části
je popsán XML manažer, který byl navržen v předchozí kapitole.

\section{Třídy, vlastnosti a jejich vztahy}
Jak bylo předesláno v předchozí kapitole, při tvorbě byl dáván velký důraz na modularitu jednotlivých součástí programu.
V praxi to znamená, že jednotlivé komunikační jednotky brány nemají tušení, jak ta druhá provádí konkrétní
operace, ale jediným spojením jsou přesně definované struktury obsahující veškeré nutné informace.

Strukturu programu tvoří čtyři třídy:
\begin{itemize}
	\item \textbf{SnmpXmlGate} - hlavní třída, která ovládá běh programu. Při spouštění inicializuje veškeré datové struktury,
	připraví oba komunikační moduly a předá jim řízení.
	\item \textbf{SnmpModule}  - komunikační modul pro zpracování veškerých SNMP požadavků. Plní dvě funkce. První je získávání dat
	z agentů v reakci na dotaz od manažera (resp. XML modulu). Druhou je zpracování asynchronních událostí (Trap, Notification), které
	se na agentech vyskytnou a získané informace zašle definovaným manažerům.
	\item \textbf{Mib2Xsd}	   - transformační třída pro překlad MIB databáze do XML schématu. Netvoří však jenom tento popis, ale zároveň 
	i vytváří finální XML dokument, se kterým pak hlavní program pracuje.
	\item \textbf{XmlModule}   - druhý komunikační modul pro zpracování XML požadavků. Tato třída je vlastní implementací
	navrženého XML protokolu a je pro tuto práci stěžejní.
\end{itemize}

Vzájemné vztahy těchto tříd jsou vyjádřeny na obrázku \ref{obr_impl_vztahy_trid}.

%\begin{figure}[htp]
%	\begin{center}
%		\includegraphics[width=15cm]{obrazky/04_obecne_schema.png}
%		\caption{Schéma navrhovahého systému}
%		\label{obr_impl_vztahy_trid}
%	\end{center}
%\end{figure}
%TODO: obrazek plus nejake vysvetleni ke vztahum (zahrnout i komunikaci apod)



\section{Popis výkonného cyklu}
Detailní popis funkcí jednotlivých tříd je součástí popisu fungování protokolové brány. Jak bylo napsáno v předchozí kapitole,
v rámci běhu programu je několik přesně definovaných fází, ve kterých se systém může nacházet. Tyto jsou vyobrazeny na obrázku \ref{obr_an_beh_programu}.
Některé z těchto fází byly oproti návrhu lehce pozměněny, či rozšířeny, dle potřebného rozsahu splňovaných funkcí.

\subsection{Inicializační fáze}
%TODO: priloha o instalaci programu
%TODO: priloha struktura konfiguracniho souboru s examplem
Je důležité připomenout, že již v návrhu bylo stanoveno, jako optimální řešení, že program bude běžet jako démon. Tomu odpovídá
i manipulace s programem. Byl vytvořen spouštěcí skript, který ovládá běh programu - spouští, zastavuje či restartuje (popis instalace,
struktura spouštěcího skriptu a použití je v příloze xx).
Jediným vstupním parametrem systému je konfigurační soubor, který je formátován jako dokument XML (přesná struktura souboru se všemi 
povolenými a nutnými elementy je v příloze yy). Tento obsahuje veškeré nutné informace pro běh systému.

\subsubsection*{Konfigurační soubor}
Informace jsou strukturovány do elementů, kdy každý popisuje jedno spravované zařízení. Každé zařízení je nutné identifikovat
unikátním číslem, které bude použito později při komunikaci s manažery. Dále je nutné specifikovat SNMP přístupové informace
k agentovi:
\begin{itemize}
	\item IP adresu či url
	\item verzi SNMP protokolu
	\item MIB báze, které popisují zařízení
	\item přístupová hesla (community string) pro zápis a čtení
\end{itemize}

Poslední položkou, která je čistě volitelná, je možné definovat manažery, kteří obdrží informace o asynchronních událostech. Každý 
manažer je definován IP adresou či url a portem, na který se budou zasílat jednotlivé zprávy.

Specifickým elementem je definice samotné brány. Tento element obsahuje veškeré informace výše popsány, ale jsou přidány ještě povinné elementy:
\begin{itemize}
	\item \textbf{logFile} - identifikuje soubor, do kterého budou ukládány veškeré textové výstupy.
	\item \textbf{snmp}    - obsahuje informace o cestě k souborům definujícím MIB databáze a portu, na kterém budou poslouchány asynchronní události
	\item \textbf{xml}     - definuje xml modul systému. Je zde verze XML protokolu; přístupová práva pro čtení a zápis; cesta k ukládání XSD popisu zařízení;
	porty pro poslouchání požadavků a odesílání odpovědí.
	\item \textbf{security} - obsahuje případné informace o certifikátu a klíči použitém při přístupu přes protokol HTTPS.
\end{itemize}

\subsection*{Zpracování a ověření informací}
Poté, co je bezchybně zpracován vstupní soubor, je přistoupeno ke zpracování informací ohledně jednotlivých zařízeních. Tuto fázi má
na starosti třída \textit{SnmpModule}. Nejprve je ověřeno, jestli dané zařízení vůbec funguje a je na něm přítomen SNMP agent. Odešle
se tedy základní požadavek a je očekávána jakákoliv odpověď. Když je agent aktivní, je zařazen do seznamu spravovaných zařízení. Jinak
systém tuto položku vynechá a nebude ji dále brát v úvahu. 

Následuje zpracování seznamu MIB souborů, které popisují nabízené informace. Tyto soubory je nutné vlastnit a přiložit je do dříve specifikovaného
adresáře, odkud si je systém načte a později zpracuje. Jestli všechny soubory existují a jsou načteny, zařízení je finálně "odsouhlaseno".

%TODO: diagram znazornujici pr;beh inicializacni faze
%\begin{figure}[htp]
%	\begin{center}
%		\includegraphics[width=15cm]{obrazky/04_obecne_schema.png}
%		\caption{Schéma navrhovahého systému}
%		\label{obr_impl_transformace_schema}
%	\end{center}
%\end{figure}
%TODO: obrazek schematickeho vyjadreni transformace

Po zpracování údajů o všech zařízeních, inicializační fáze končí a systém započne fázi transformační.

\subsection{Transformační fáze}
V této chvíli přebírá úlohu třída \textit{\textbf{Mib2Xsd}}, která se stará o převod MIB popisu databáze do XSD formátu. Pro manipulaci se SNMP je
použita knihovna \textit{\textbf{net-snmp}}. %TODO: citace zdroje netsnmp

Samotný převod a generování XSD popisu je rekurzivní sestup po jednotlivých uzlech virtuálního stromu MIB databáze. Proces je schématicky naznačen
na obrázku \ref{obr_impl_trasformace_schema}.
%\begin{figure}[htp]
%	\begin{center}
%		\includegraphics[width=15cm]{obrazky/04_obecne_schema.png}
%		\caption{Schéma navrhovahého systému}
%		\label{obr_impl_transformace_schema}
%	\end{center}
%\end{figure}
%TODO: obrazek schematickeho vyjadreni transformace

Pro každé zařízení systém načte všechny specifikované MIB soubory a započne s transformací. Proces se řídí pravidly, které byly definovány v kapitole \ref{kap_xml}.
Jednotlivé uzly jsou popsány svým typem, který obsahuje jejich unikátní identifikační číslo (OID), typ dat a přístupová práva. Samotné uzly jsou pak zařazeny
do stromové hierarchie v rámci dokumentu.

Informace vztahující se k jednotlivým zařízeními jsou generovány do oddělených souborů. Samostatný soubor
pak obsahuje schéma brány a pouze identifikaci spravovaných zařízení. Tento systém byl vytvořen pro ušetření komunikační zátěže mezi manažerem a bránou.
Bližší popis výhod tohoto přístupu dále v této kapitole.

Oproti navrženému systému mapování MIB do XSD (\cite{macejko_dipl}) byla vynechána nutnost použítí prostorů jmen pro jednotlivé MIB databáze. Vzhledem
k tomu, že každý element je popsán unikátním jménem a OID, nedojde při zpracování požadavků ke konfliktu.

Abychom ušetřili procesorový čas, je paralelně s generováním schématu vytvářen i XML dokument, který za běhu slouží k vyhledávání a ostatním potřebným
operacím. Oproti schématu vypadá element specifikující uzel v MIB databázi následovně:
%TODO obrazek struktury XML dokumentu
%\begin{figure}[htp]
%	\begin{center}
%		\includegraphics[width=15cm]{obrazky/04_obecne_schema.png}
%		\caption{Schéma navrhovahého systému}
%		\label{obr_impl_transformace_schema}
%	\end{center}
%\end{figure}
%TODO: obrazek schematickeho vyjadreni transformace

Tento dokument je pak spravován a využíván třídou \textit{\textbf{XmlModule}}.


\subsection{Komunikační moduly}
Po úspěšném zpracování a trasformaci MIB databází je možné přistoupit k inicializaci komunikačních jednotek a otevření síťových spojení k bráně.
V této fázi předává třída \textit{ \textbf{SnmpXmlGate} } řízení třídám \textit{ \textbf{SnmpModule} } a \textit{ \textbf{XmlModule} }. Tyto
moduly pak zajišťují kompletní provoz protokolové brány.

\subsubsection*{XmlModule}
\label{kap_impl_xmlmod}
Dle návrhu programu v kapitole \ref{kap_analyza} bylo uvedeno, že pro komunikaci mezi bránou a manažery bude použit HTTP server zabudovaný do
aplikace, abychom měli větší kontrolu nad posílanými daty. Proto byl použit nejvhodnější kandidát - microHTTP server %TODO cite reference na libmicrohhtpd.
Tento server je napsán v jazyce C a velice dobře posloužil našemu účelu. 

Jednotlivá spojení jsou zpracovávána samostatnými vlákny, aby bylo zaručeno co nejrychlejší odbavení požadavků. Funkční cyklus vlánka je znázorněn
na schématu \ref{obr_impl_http_vlakno}. 
%\begin{figure}[htp]
%	\begin{center}
%		\includegraphics[width=15cm]{obrazky/04_obecne_schema.png}
%		\caption{Schéma navrhovahého systému}
%		\label{obr_impl_transformace_schema}
%	\end{center}
%\end{figure}
%TODO: obrazek schematickeho vyjadreni transformace

Nejprve je zpracována HTTP zpráva. Systém operuje pouze s HTTP zprávami typu POST. Jakékoliv GET zprávy jsou zahazovány a zpět klientovi je zasláno
chybové upozornění. Poté je na data použit XML parser, který zpracuje požadavek a ujistí se, že je ve formátu specifikovaném v kapitole \ref{kap_xml}.
Všechny zprávy jsou zpracovávány najednou, aby byla zaručena integrita dat a "atomicita" operací. Více v kapitole \ref{kap_impl_fronta_atomicita}. 
Přesné zpracování jednoho požadavku je popsáno dále v této kapitole.

Pro komunikaci vláken s druhým modulem byly vytvořeny fronty požadavků (pro směr od XML modulu k SNMP modulu)
a jedna odpovědní fronta, ze které se poté vybírají vyřízené požadavky.
Když jsou všechny zprávy v pořádku zpracovány, jsou odeslány do fronty k příslušnému monitorovanému zařízení, nebo jsou rovnou zařazeny
jako odpovědi (obsahují-li chybu, která znemožňuje její další zpracování). Zařazení do fronty k vláknu má na starosti SNMP modul a zároveň
i on probouzí příslušné vlákno ke zpracování požadavků.

Vlákno se následně přesune do čekajícího cyklu, kde kontroluje odpovědní frontu na svoje požadavky do doby, než se dostaví všechny odpovědi. 
Poté vygeneruje odpovědní zprávu a zašle manažerovi. Tím jeho životní cyklus prakticky končí. Je na serveru, jak s vlákny poté naloží a jak 
je recykluje. Takzvané zamrznutí a nekonečné čekání vlákna je ošetřeno maximální prodlevou pro jeden požadavek v SNMP modulu. Tím je zaručeno, že
manažer určitě dostane odpověď.

\subsubsection*{Jednotka komunikace}
Než přistoupíme k popisu zpracování jednotlivých XML zpráv, je důležité popsat, jak spolu komunikují oba moduly. Z navrženého protokolu jasně 
vyplývá, že XML manažer komunikuje s bránou tak, jako by to byl samotný XML agent. O protokolu SNMP nesmí vědět. Proto byl již při návrhu
brán zřetel na striktní oddělení těchto dvou protokolů. Pro komunikaci byla navržena struktura, která v sobě obsahuje veškeré informace o
požadavku a odpovědi, ale je protokolově neutrální.

Důležité položky struktury, které oba moduly využívají jsou:
\begin{itemize}
	\item \textbf{typ zprávy} - definuje, jaký požadavek byl vyslán (Get, Set). Dle toho se pak vytvářejí SNMP požadavky na agenta.
	\item \textbf{seznam uzlů} - seznam dvojic (jméno, hodnota), které manažer požaduje.
	\item \textbf{chyba} - identifikace a textová reprezentace chyby, která v celém zpracujícím procestu nastala.
\end{itemize}

Tato struktura pak obsahuje informace o jednom požadavku (Get, Set, Subscribe), který byl v XML zprávě obsažen. 


\subsubsection*{SnmpModule}
Komunikace se SNMP agentama je založen na systému front. Každé monitorované zařízení má na straně protokolové brány k dispozici jedno obslužné vlákno
a frontu zpráv. Schématicky naznačeno na obrázku \ref{obr_impl_snmp_komunikace}. Zpracování a vytváření SNMP zpráv se děje pomocí knihovny \textbf{net-snmp}.
%\begin{figure}[htp]
%	\begin{center}
%		\includegraphics[width=15cm]{obrazky/04_obecne_schema.png}
%		\caption{Schéma navrhovahého systému}
%		\label{obr_impl_transformace_schema}
%	\end{center}
%\end{figure}
%TODO: obrazek schematickeho vyjadreni transformace

Práce komunikačních vláken probíhá v "nekonečném" cyklu (znázorněném na obrázku \ref{obr_impl_komun_vlakno}). Vstupní informací pro vlákno je identifikační
číslo monitorovaného zařízení, ke kterému patří. S touto informací získá přístup k frontě zpráv a dalším potřebným strukturám, příslušející pouze danému agentovi.
V cyklu se pak opakuje několik dílčích bloků stále dokola.
%\begin{figure}[htp]
%	\begin{center}
%		\includegraphics[width=15cm]{obrazky/04_obecne_schema.png}
%		\caption{Schéma navrhovahého systému}
%		\label{obr_impl_transformace_schema}
%	\end{center}
%\end{figure}
%TODO: obrazek schematickeho vyjadreni transformace

Vlákno zjistí, jestli v příslušné frontě je přítomen nějaký požadavek na vyřízení. Když není, vlákno se uspí. Jinak vyjme všechny dotazy, tím frontu vyprázdní a
požadavky zpracuje. Tento funkční blok závisí na typu zprávky, která se má vyřídit. 

Zprávy Get a Set, které mají za úkol dostat nebo nastavit hodnotu koncového uzlu stromu, plně korespondují se SNMP. Je vytvořen SNMP packet, do kterého je vloženo
heslo (community string) na základě přístupových práv, které manažer má, je naplněno daty, které se mají nastavit či získat a dotaz je odeslán.

Po přijetí odpovědi jsou data opět z paketu vyjmuta, upravena do požadovaného formátu v rámci komunikační struktury. Pakliže se vyskytl error při zpracování, opět
je nastaven příznak ve struktuře, aby manažer dostal informace o chybě. Takto vygenerovaná odpověď je poslána do odpovědní fronty a je probuzeno vlákno, které na
data čeká (samotná funkce je ve správě XML modulu).

Složitější je zpracování požadavku na uzel, který není koncový a obsahuje pod sebou celý podstrom dalších uzlů (z hlediska MIB databáze). Poté je nutno opakovat
vytváření paketů a zasílání dotazů ve formě SNMP zprávy GetNextRequest. Až poté, co jsou získána všechna data, je navrácena odpověď klientskému vláknu v XML modulu.


\subsection{Zprávy}
Zpracování jednotlivých druhů zpráv se odvíjí od navrženého modelu. Následuje popis chování systému při přijetí jednotlivých požadavků.

\subsubsection*{Discovery}
Jediným omezením při zpracování požadavku je verze XML protokolu, kterou manažer posílá ve zprávě. Když se rozchází s nastavenou 
verzí v systému, odpovědí je zpráva Publication a chybové hlášení.

Jedním z implementačních požadavků bylo též snížit množství dat mezi manažerem a bránou. Předpokládáme, že brána spravuje
desítky SNMP zařízení. Celý dokument, popisující všechny informace, který je odpovědí na tento požadavek by pak nabyl
obrovských rozměrů. Pro zlepšení protokolu tak systém zašle manažerovi pouze hlavní popis brány s identifikací jednotlivých zařízení, bez ostatních dat.
Manažer si může vybrat, které zařízení chce spravovat a zašle novou zprávu Discovery s volitelným atributem \textbf{objectid}. Po zpracování
mu brána zašle pouze XSD dokument popisující právě to jedno zařízení.

\subsubsection*{Get a Set}
Zpracování těchto požadavků prochází několika fázemi v rámci samotného XML modulu.

Nejprve je nutné zjistit, jestli má manažer právo tyto operace provádět. Heslo, které se zasílá v elementu \textit{message}, určuje, jestli je
povoleno čtení, či zápis. Nastavení hesel probíhá při inicializační fázi a tyto jsou čtena z konfiguračního souboru. Celý přístupový model je
založen na dvou heslech XML protokolu - pro čtení a zápis. Každé SNMP zařízení má pak definovány opět dvě hesla, která jsou odlišná od těch prvních jmenovaných.
Systém pak na základě dodaného XML hesla zjistí, jaké má uživatel práva a použije pak příslušné SNMP heslo pro požadavek. Může se stát, že mangažer zadal
špatné heslo a systém mu přístup odepře. Pak je odpovědí na tyto požadavky chybová zpráva.

Když uživatel má dostatečné oprávnění, je nutné zjistit, zda-li uzly, na které se dotazuje, opravdu existují. Dotazování probíhá pomcí jazyka XPath.
Pakliže je dotaz špatně formulován, nebo element neexistuje, systém požadavek zamítne. V druhém případě mohou nastat dvě eventuality.

Element je koncovým uzlem stromu, tj. obsahuje pouze hodnotu danou definovaným typem. Systém takový dotaz interpretuje jako jediný SNMP dotaz GET.
Když se ale jedná o kořenový uzel (obsahuje podstrom uzlů), je nutné tento dotaz interpretovat jako
SNMP zprávu GETNEXT a získat všechny hodnoty daného podstromu (o což se stará SNMP modul).

Jestli bylo požadavkem nastavení určité hodnoty, pak je součástí datové struktury jméno uzlu a příslušná hodnota.

Vygenerovaná komunikační jednotka je vložena do fronty požadavků a pak současně se všemi ostatními odeslána SNMP modulu ke zpracování.

\subsubsection*{Subscribe}
Touto zprávou se manažer upisuje k zasílání pravidelných informací o daném agentovi, nebo tyto údaje mění a maže. Každý takový záznam
se zanese do hlavního XML dokumentu, aby bylo při úpravách a mazání možné vyhledávat pomocí XPath výrazu. Reakce na přidání a úprava
vnitřních struktur je ponechána na speciální vlákno, které slouží k obsluze daných záznamů. Bližší specifikace dále v této kapitole.

Jediný problém při implementaci nastal v rámci mazání jednotlivých záznamů. Vzhledem k tomu, že v navrženém protokolu je odpovědí
manažerovi zpráva Distribution s daty, která požaduje, nebo chyba, je nutno při smazání záznamu poslat opět zprávu Distribution, ale nyní
prázdnou.

Práva k mazání a úpravám daného záznamu jsou ponechána pouze na znalosti unikátního identifikátoru. 


\subsubsection*{Změny oproti navrženému XML protokolu}
Oproti navrženému XML protokolu bylo nutno udělat několik drobných změn, abychom mohli celý systém implementovat. Jedná se pouze o
minoritní změny - přidání volitelného atributu \textbf{objectid} ke všem typům požadavků, které manažer zasílá. Je to z důvodu
identifiace monitorovaného zařízení, což by bez tohoto zásahu nebylo vůbec možné.

Abychom nezměnili dosavadní protokol, byl tento atribut zvolen jako volitelný. Jeho nepřítomnost signalizuje, že dotaz je mířen
na samotné zařízení brány, která následně přepošle požadavek na SNMP agenta běžícího na tom samém stroji.


\subsection{Fronty a atomicita požadavků}
\label{kap_impl_fronta_atomicita}
Systém předávání požadavků pomocí front vychází z navrženého protokolu (\cite{macejko_dipl}), ale byl přepracován pro potřeby
této protokolové brány. Navrhovány byly fronty, které by zpracovávaly požadavky dle priority a mamažer by si mohl říci, které
dotazy budou mít jakou prioritu. To bohužel zcela nevyhovuje požadavku zajištění posloupnosti provádění dotazů a tím pádem i 
integritě dat.

Předpokládejme situaci, kdy manažer v jedné XML zprávě zašle dotaz na hodnotu jednoho uzlu s nižší prioritou a zároveň také nastavení
jiné hodnoty s vyšší prioritou. Stalo by se, že při zpracovávání se nastavení provede dříve a manažer by se nikdy nedozvěděl starou hodnotu.
%TODO: obrazek chyby, ktera by nastala SET drive nez GET
%\begin{figure}[htp]
%	\begin{center}
%		\includegraphics[width=15cm]{obrazky/04_obecne_schema.png}
%		\caption{Schéma navrhovahého systému}
%		\label{obr_impl_transformace_schema}
%	\end{center}
%\end{figure}
%TODO: obrazek schematickeho vyjadreni transformace

Druhým problémem je "atomicita" prováděných operací. Kdyby systém zpracovával jeden požadavek z přijaté zprávy, čekal na odpověď agenta, a pak
přešel na další, mohlo by se stát, že druhý manažer by mohl do tohoto cyklu vstoupit a narušit tak výsledek celé posloupnosti operací.
%TODO: obrazek chyby, ktera by nastala SET drive nez GET
%\begin{figure}[htp]
%	\begin{center}
%		\includegraphics[width=15cm]{obrazky/04_obecne_schema.png}
%		\caption{Schéma navrhovahého systému}
%		\label{obr_impl_transformace_schema}
%	\end{center}
%\end{figure}
%TODO: obrazek schematickeho vyjadreni transformace

Proto byl systém přepracován tak, že každé monitorované zařízení má v SNMP modulu jedno obslužné vlákno a frontu požadavků (jak bylo vysvětleno 
dříve v této kapitole). Nyní se v XML modulu zpracují všechny požadavky v rámci jedné zprávy, vygenerují se příslušné komunikační struktury a ty jsou
následně všechny najednou vloženy do příslušné fronty požadavků. Tato funkce je ponechána na SNMP modulu, který zajistí, pomocí systému výlučného přístupu
k dané frontě, aby všechny zprávy tvořily souvislý blok, který bude sekvenčně zpracován.

Je možné, že uživatel v rámci jedné zprávy zašle dotazy na více zařízení. Každé takové zařízení obdrží svůj blok dotazů individuálně.
%TODO: obrazek chyby, ktera by nastala SET drive nez GET
%\begin{figure}[htp]
%	\begin{center}
%		\includegraphics[width=15cm]{obrazky/04_obecne_schema.png}
%		\caption{Schéma navrhovahého systému}
%		\label{obr_impl_transformace_schema}
%	\end{center}
%\end{figure}
%TODO: obrazek schematickeho vyjadreni transformace

Předávání odpovědí ze SNMP do XML modulu slouží pouze jediná odpovědní fronta. Do ní se vkládají postupně vyřízené požadavky ze všech zařízení.
Vždy, když je vložena odpověď, jsou probuzena klientská vlákna, zkontrolují frontu a případně si vyberou jim určené odpovědi. Zde by se mohlo zdát,
že bude narušena posloupnost jednotlivých odpovědí v případě, ze komunikujeme s více zařízeními. Tento fakt je ošetřen unikátním identifikátorem každého
požadavku v zasílané zprávě. Odpovědi poté obsahují stejný identifikační řetězec, díky kterému pak manažer pozná, k jaké zprávě se vztahuje obdržená hodnota.

\subsection{Periodické zasílání informací}
\subsection{Asynchronní události}


\section{Paměťové optimalizace}

\section{Použité knihovny}

\section{XML Manažer}
