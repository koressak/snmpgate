\chapter{Testování}
\label{kap_testovani}
Cílem této práce bylo vytvoření funkčního prototypu protokolové brány v jazyze C++. Proto
veškeré testování bude zaměřeno na ověření funkce a prověření reakce-schopnosti na jednotlivé
manažerské požadavky. Z podstaty práce vyplývá, že krom výše zmíněných parametrů není nutné provádět
jakákoliv měření, jelikož povaha programu není založena na rychlosti odpovědí, ani jiných časových parametrech.

\subsection*{Testovací systém}
Síť agentů spojená s bránou, na které bylo prováděno měření, se sestává ze dvou autonomních systémů.
Prvním je notebook, na kterém je nainstalován SNMP agent. Druhým systémem je zařízení brány, na kterém
kromě samotné protokolové aplikace běží též SNMP agent.

Testy se sestávají z ověření funkce pomocí odeslání jednotlivých požadavků dle navrženého protokolu a obdržení odpovědí.
V každé zprávě je nejprve uvedeno, co manažer odeslal a následuje výpis, co brána odpověděla.

\subsection*{Test zprávy Discovery}
Dotaz:
\begin{verbatim}
<message password="zapis">
  <discovery protocolVersion="1"  msgid="1"  />
</message>
\end{verbatim}

Zpráva má vyvolat odpověď obecného popisu a seznamu všech zařízení, které brána spravuje. Schéma
jednotlivých zařízení je natolik obsáhlé, že jej není možné do této práce vložit.

\newpage
Odpověď (výstup zkrácen kvůli lepší čitelnosti):
\begin{verbatim}
<message>
  <publication msgid="1">
    <info> <xpath>1.0</xpath> </info>
    <dataModel>
      <xsd:schema attributeFromDefault="unqualified" ...>
	  ...
      <xsd:element name="devices">
        <xsd:complexType>
          <xsd:sequence>
            <xsd:element id="0" name="device">
              <xsd:complexType>
                <xsd:sequence>
                  <xsd:element name="name">SNMP Gate</xsd:element>
                  <xsd:element name="description">
                  Popis brany
                  </xsd:element>
                  <xsd:element name="data"/>
                  <xsd:element name="notifications"/>
                </xsd:sequence>
              </xsd:complexType>
            </xsd:element>
            <xsd:element id="1" name="device">
              <xsd:complexType>
                <xsd:sequence>
                  <xsd:element name="name">notebook</xsd:element>
                  <xsd:element name="description">
                  Domaci notebook
                  </xsd:element>
                  <xsd:element name="data"/>
                  <xsd:element name="notifications"/>
                </xsd:sequence>
              </xsd:complexType>
            </xsd:element>
          </xsd:sequence>
        </xsd:complexType>
      </xsd:element>
    ...
    </xsd:schema>
  </datamodel>
  </publication>
</message>
\end{verbatim}
\newpage

\subsection*{Test zprávy Get}
Dotážeme se na popis systému jednotlivých zařízení.
\begin{verbatim}
<message password="zapis">
  <get  msgid="1"  objectId="0"  > 
    <xpath>/iso/org/dod/internet/mgmt/mib-2/system/sysDescr</xpath>
  </get>

  <get  msgid="2"  objectId="1"  > 
    <xpath>/iso/org/dod/internet/mgmt/mib-2/system/sysDescr</xpath>
  </get>
</message>
\end{verbatim}

Odpověď:
\begin{verbatim}
<message>
  <response msgid="1">
    <value>
      Linux zoo 2.6.26-1-amd64 #1 SMP Sat Jan 10 19:55:48 UTC 2009 x86_64
    </value>
  </response>

  <response msgid="2">
    <value>
      Linux zoo 2.6.26-hrosi-jadro #4 SMP Fri Nov 21 19:19:38 CET 2008 i686
    </value>
  </response>
</message>
\end{verbatim}

\subsection*{Test zprávy Subscribe}
V intervalech 5 vteřin se chceme dotazovat na počet TCP paketů, které přišly na vstup zařízení číslo 1 (domácí notebook).
\begin{verbatim}
<message password="zapis">
  <subscribe  msgid="1"  objectId="1"  frequency="5"  >
    <xpath>/iso/org/dod/internet/mgmt/mib-2/tcp/tcpInSegs</xpath>
  </subscribe>
</message>
\end{verbatim}

Odpovědí nám jsou data, čímž brána potvrzuje přijetí a bezchybnost zápisu. Ihned manažer zapíná HTTP server,
který poslouchá na definovaném protu a čeká na došlé informace (v našem případě jsme nechali přijmout
5 zpráv).
\begin{verbatim}
<!-- Prvni odpoved od serveru -->
<message>
  <distribution msgid="1" distrid ="1">
    <value>
      Counter32: 6114
    </value>
  </distribution>
</message>

------------------
Starting server for distributions
-----------------
Data received: 
<message context="">
  <distribution msgid="1" distrid ="1">
    <value>
      Counter32: 6114
    </value>
  </distribution>
</message>
-----------------------------
------------------------------
Data received: 
<message context="">
  <distribution msgid="2" distrid ="1">
    <value>
      Counter32: 6114
    </value>
  </distribution>
</message>
-----------------------------
------------------------------
Data received: 
<message context="">
  <distribution msgid="3" distrid ="1">
    <value>
      Counter32: 6114
    </value>
  </distribution>
</message>
-----------------------------
------------------------------
Data received: 
<message context="">
  <distribution msgid="4" distrid ="1">
    <value>
      Counter32: 6114
    </value>
  </distribution>
</message>
-----------------------------
------------------------------
Data received: 
<message context="">
  <distribution msgid="5" distrid ="1">
    <value>
      Counter32: 6114
    </value>
  </distribution>
</message>
-----------------------------
\end{verbatim}

\subsection*{Test zprávy Event}
V této fázi je manažer nastartován pouze s HTTP serverem poslouchajícím na definovaném portu.
Očekává se přijetí zprávy Event, nebo ukončení programu. V našem případě simulujeme
nastalou událost pomocí aplikace \verb|snmptrap|. Simulujeme výpadek jednoho síťového spojení (linkDown), což
je jedna ze standardních událostí (generic trap).

Přijatá zpráva manažerem:
\begin{verbatim}
------------------
Starting server for notifications
------------------------------
Data received:
<message>
  <event msgid="1"
    senderid="0" 
    eventSpec="device/notifications/linkDown">
      <data>
      </data>
  </event>
</message>
-----------------------------
\end{verbatim}

