\chapter{Úvod}
Správa velkých počítačových sítí je v dnešní době naprosto samozřejmým úkolem většiny administrátorů. Velké množství spravovaných sítí se neomezuje pouze na lokální prostředí dané firmy či instituce. Může být naopak rozprostřena 
v rámci jednoho města, státu či dokonce několika států najednou. Efektivní spravování takovéto komunikační infrastruktury je úkolem velice náročným.

Jedním z protokolů, který takovouto vzdálenou správu umožňuje, je SNMP. Na jeho základě bylo vybudováno bezpočet aplikací, které mají za úkol sledovat provoz na síti, zatížení určitého systému a v neposlední řadě umožnit administrátorovi
vzdálenou správu daného přepínače, routeru či pracovní stanice.

Protokol SNMP byl navržen v dřívějších dobách a nemusí plně vyhovovat dnešním požadavkům, až už na bezpečnost nebo efektivní využití přenosových médií. Pan Ing. Peter Macejko se ve své diplomové práci (\cite{macejko_dipl}) zaobíral použitím 
technologií XML a návrhu protokolu, který by umožňoval minimálně stejnou funkcionalitu jako protokol SNMP a tento zefektivnil.

Tato práce se zaobírá vytvořením protokolové brány, která by umožnila použít navržený XML protokol ke správě strojů, které stále používají protokol SNMP. Cílem je vytvořit softwarový produkt, který bude plnit úkol prostředníka mezi správcem a spravovaným
strojem. Hlavními problémy jsou implementace navrženého XML protokolu a spojení jej s několika verzemi protokolu SNMP. Důležitým aspektem vývoje je i orientace na snížení nároků na operační paměť. Proto bude v každé části programu kladen důraz na efektivní správu
datových struktur.

V kapitole \ref{kap_snmp} bude podrobně popsán protokol SNMP, jeho komunikační struktury a typy zpráv.

Kapitola \ref{kap_xml} se zabývá rozborem navrženého XML protokolu. 

Analýza systému a diskuse o možných směrech implementace bude popsána v kapitole \ref{kap_navrh_systemu}.

V kapitole \ref{kap_implementace} bude probrána detailní funkčnost implementovaného programu.

Ověření funkce a další testování systému bude popsáno v kapitole \ref{kap_testovani}.

